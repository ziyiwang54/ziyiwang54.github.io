% ------------------------------------------------------------------
% Ziyi Wang - Resume
% One-page concise technical resume
% ------------------------------------------------------------------
% Compilation: xelatex (preferred) or pdflatex
% ------------------------------------------------------------------
\documentclass[10pt,letterpaper]{article}

% ---------- Packages ----------
\usepackage[letterpaper,margin=0.6in,top=0.55in,bottom=0.55in]{geometry}
\usepackage{setspace}
\usepackage{enumitem}
\usepackage{fontawesome5}
\usepackage[bookmarks=false,colorlinks=true,urlcolor=black,linkcolor=black,pdfauthor={Ziyi Wang},pdftitle={Resume - Ziyi Wang}]{hyperref}
\usepackage{titlesec}
\usepackage{array}
\usepackage{xcolor}
\usepackage{ifxetex,ifluatex}
\ifxetex
  \usepackage{fontspec}
  \setmainfont{Latin Modern Roman}
\fi
\ifluatex
  \usepackage{fontspec}
  \setmainfont{Latin Modern Roman}
\fi

% ---------- Hyperref ----------
\urlstyle{same}

% ---------- Formatting Tweaks ----------
\setlength{\parindent}{0pt}
\setlength{\parskip}{0pt}
\setlist[itemize]{left=0.9em,label=\textbullet,topsep=2pt,itemsep=2pt,parsep=0pt,partopsep=0pt}
\renewcommand{\baselinestretch}{1.05}

% ---------- Section Formatting ----------
\titleformat{\section}{\bfseries\Large}{}{0pt}{}[\vspace{-2pt}\rule{\linewidth}{0.4pt}\vspace{-4pt}]
\titleformat{\subsection}{\bfseries}{}{0pt}{}
\titlespacing*{\section}{0pt}{6pt}{2pt}
\titlespacing*{\subsection}{0pt}{4pt}{2pt}

% ---------- Custom Commands ----------
\newcommand{\name}[1]{\centerline{\Huge \bfseries #1}}%
\newcommand{\contact}[1]{\vspace{4pt}\centerline{#1}\vspace{4pt}}%
\newcommand{\headingrow}[4]{\textbf{#1} \hfill {#2}\\\emph{#3} \hfill {#4}}%
\newcommand{\role}[4]{\textbf{#1} \hfill {#2}\\\emph{#3} \hfill {#4}}%
\newcommand{\skill}[2]{\textbf{#1:} #2\\}

% Avoid page number
\pagestyle{empty}

\begin{document}

% ---------- Header ----------
\name{Ziyi Wang}
\contact{\faPhone\; (217) 974-6103 \;|\; \faEnvelope\; \href{mailto:ziyiwangandd@outlook.com}{ziyiwangandd@outlook.com} \;|\; Champaign, IL 61820}

% ---------- Education ----------
\section*{Education}
\headingrow{University of Illinois at Urbana-Champaign}{Urbana, IL}{B.S. Computer Engineering (Expected May 2026)}{GPA: \textbf{3.90}/4.00}
Dean's List. 
\\
\textbf{Core Coursework}: Computer Systems Engineering, Control Systems, Robot Dynamics and Control, Deep Learning for Computer Vision, Digital Signal Processing.

% ---------- Research Experience ----------
\section*{Research Experience}
\role{Vision-Based Autonomous Perching}{May 2025 -- Sep 25}{Undergraduate Researcher, Advanced Control Research Lab (Advisor: Prof. Naira Hovakimyan)}{UIUC}
\begin{itemize}
  \item Developed vision-based algorithms for autonomous perching capabilities enabling precise landing and attachment on various surfaces and structures.
  \item Implemented computer vision pipelines for real-time detection and pose estimation of perching targets using camera/sensor specifications.
  \item Designed and validated control algorithms for approach trajectories and contact dynamics during perching maneuvers.
\end{itemize}

\role{Safety-Critical Control via Online System ID \& Control Barrier Functions}{Aug 2024 -- May 2025}{Undergraduate Researcher, RoboDesign Lab (Advisor: Prof. Joao Ramos)}{UIUC}
\begin{itemize}
  \item Engineered an end-to-end RGB-D perception pipeline (capture $\rightarrow$ point cloud filtering $\rightarrow$ feature extraction) delivering robust shape/pose priors to accelerate controller inertia estimation for humanoid SATYRR.
  \item Implemented center snapping and geometry estimation by segmenting point clouds from depth data; produced consistent results across varying lighting/occlusion conditions.
  \item Customized NVIDIA Jetson Orin Nano environment (device tree overlays, driver configuration) enabling real-time sensing and motor interfacing on embedded platform.
\end{itemize}

\role{Weakly-Supervised Traversability Prediction}{Jan 2024 -- May 2024}{Undergraduate Researcher, Human-Centered Autonomy Lab (Advisor: Prof. Katherine Driggs-Campbell)}{UIUC}
\begin{itemize}
  \item Tuned MPPI controller (ROS Noetic / Gazebo) and packaged improvements into a reusable ROS module for traversability-aware planning.
  \item Collected multi-environment datasets (campus + field) and prepared weakly-labeled training corpus; trained a neural model to refine sampling for safer navigation.
  \item Investigated backtracking strategies for low-supervision systems in semi-structured agricultural terrains.
  \item Self-studied and integrated ROS, PyTorch, and Gazebo toolchains for rapid experimentation.
\end{itemize}

% ---------- Publications ----------
\section*{Publications}
\begin{itemize}
  \item Han, Z., Mo, B., Cheng, S., Wang, R., \textbf{Wang, Z.}, Gao, J., Pham, H. T., Ho, V. A., and Hovakimyan, N. ``Perch: A Vision-Based Approach for Autonomous Perching'' \emph{IEEE International Conference on Robotics and Automation (ICRA)}, 2026. \textbf{(Under Review)}
\end{itemize}

% ---------- Industry Experience ----------
\section*{Industry Experience}
\role{Software and DevOps Engineer Intern}{May 2025 -- Aug 2025}{Laplace AI Lab Inc.}{}
\begin{itemize}
\item Developed an \textbf{Approval MCP tool} enabling structured request/response workflows across LLM agents, providing fine-grained approval and rejection controls for multi-agent collaboration.

\item Implemented \textbf{MCP server logic} (FastAPI / Node.js) to handle agent requests, approval queues, and decision logging, reducing unsafe or redundant agent actions.

\item Integrated the tool with \textbf{Dockerized agent infrastructure} and external services (databases, APIs), ensuring seamless interoperability in multi-agent pipelines.

\item Enhanced \textbf{safety and reliability} of agent execution by introducing human-in-the-loop checkpoints, preventing unauthorized or unintended actions in production workflows.

\end{itemize}
\role{Embedded Software Engineer Intern}{May 2024 -- Jun 2024}{Shenzhen Weijing Power Intelligence Co., Ltd.}{Shenzhen, China}
\begin{itemize}
  \item Trained and quantized YOLOv8-based visual recognition model for sensor identification; achieved \~70\% confidence after deployment to microcontroller + TPU (RKNN toolkit).
  \item Curated and organized 8GB hand gesture dataset for future training and parameter tuning cycles.
  \item Fused inertial sensor data (multi-axis acceleration) to compute orientation via quaternion and rotation matrix pipelines for embedded control modules.
\end{itemize}

% ---------- Leadership & Projects ----------
\section*{Leadership \& Projects}
\role{Course Assistant, ECE 391: Computer Systems Engineering}{Jan 2025 -- Present}{Department of Electrical and Computer Engineering}{UIUC}
\begin{itemize}
  \item Participated in the \textbf{design and revision of course materials and machine problems (MPs)} covering systems programming and operating systems topics.
  \item Held office hours and provided hands-on debugging support in \textbf{C}, \textbf{riscv}, and Unix tooling; clarified concepts such as processes, scheduling, filesystems, and concurrency.
  \item Assisted with grading and feedback to improve clarity, consistency, and student outcomes across sections.
\end{itemize}

\role{Embedded Team Lead, Illini RoboMaster}{Jan 2023 -- Present}{Student Robotics Organization}{UIUC}
\begin{itemize}
  \item Lead embedded systems development for autonomous combat robots: microcontroller firmware (FreeRTOS, CMSIS-RTOS), CAN bus communication, and Jetson-based vision/navigation integration.
  \item Established structured codebase standards (task scheduling, message abstraction) improving feature iteration speed and on-field debug efficiency.
\end{itemize}

\textit{Selected Projects:}
\begin{itemize}
  \item \textbf{Teleoperation Control for 6-DOF Robotic Arm (May 2024 -- Aug 2024):} Implemented custom controller using 6 bus servos; packed joint angles into 16-channel PPM frame over USART bridging to DBUS remote; integrated ROS MoveIt for motion/trajectory planning.
  \item \textbf{Swerve Chassis Standard Robot (Jun 2023):} Optimized control algorithm, added current-based CAN callbacks, and built real-time self-check routines for rapid on-field hardware diagnostics.
  \item \textbf{Project Management System (Aug 2025 -- Sep 2025):} Developed a web-based project management tool using React and Node.js to streamline task tracking and team collaboration for student organization, improving the project tracking efficiency.
\end{itemize}

% ---------- Honors & Awards ----------
\section*{Honors \& Awards}
\begin{itemize}
  \item 2nd Place (1v1 Confrontation), RoboMaster University League North America 2023 (Top 10\%).
  \item 2025 - 2026 \textbf{Yunni and Maxine Pao Memorial Scholarship}, Grainger Engineering University of Illinois. (Awarded to only 9 students in the college annually) (06/2025)
  \item 2025 - 2026 \textbf{The John and Sheila Woythal Scholarship} of the Department of Electrical and Computer Engineering, University of Illinois. (Awarded to 10 students in the department annually) (05/2025)
\end{itemize}


% ---------- Skills ----------
\section*{Skills}
\skill{Programming}{C, C++, Python, Java, MATLAB, LaTex}
\skill{AI / CV}{YOLO, PyTorch, scikit-learn}
\skill{Robotics / Control}{ROS, PID control}
\skill{Embedded / MCU}{STM32, Arduino, FreeRTOS, CMSIS-RTOS V2, CAN, USART}
\skill{HDL / Digital Design}{VHDL, Verilog}
\skill{Hardware}{Oscilloscope, Multimeter}
\skill{Tools}{Git, Docker, FastMCP}
\skill{OS}{Linux (Ubuntu/Jetson), shell, systemd/udev, networking, device tree overlays, driver configuration}
\skill{Languages}{English (Native), Mandarin (Native), Cantonese (Conversational)}



\end{document}
